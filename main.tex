\documentclass[12pt,-letter paper]{article}
\usepackage{gvv}
\begin{document}
\begin{enumerate}
\item An acute-angled triangle $ABC$ has orthocentre $H$. The circle passing through $H$ withcentre the midpoint of $BC$ intersects the line $BC$ at $A1$ and $A2$. Similarly, the circle passing through $H$ with centre the midpoint of $CA$ intersects the line $CA$ at $B1$ and $B2$,and the circle passing throughH with centre the midpoint of $AB$ intersects the line $AB$ at $C1$ and $C2$. Show that $A1$, $A2$, $B1$, $B2$,$C1$, $C2$ lie on a circle.
\item $\brak a$ Prove that 
$\frac{x^2}{{(x-1)}^{2}}$+$\frac{y^2}{{(y-1)}^{2}}$+$\frac{z^2}{{(z-1)}^{2}} \geq 1$ \\ 
for all real numbers $x, y, z$, each different from $1$, andsatisfying $xyz = 1$.\\

	$\brak b$ Prove that equality holds above for infinitely many triples of rational numbers $x, y, z$, each different from $1$, and satisfying $xyz=1$.
\item Prove that there exist infinitely many positive integers $n$ such that $n^2+1$ has a prime divisor which is greater than $2n+\sqrt2n$.
\item Find all functions $f$ : $\brak {0,\infty}$ $\rightarrow$ $\brak 0$, $\infty$  so, (f is a function from the positive real numbers to the positive real numbers) such that \\
	$\frac{(f(w))^2 + (f(x))^2} {f(y)^2 + f(z)^2 }$
	for all positive real numbers $w, x, y, z$, satisfying $wx$ = $yz$.
\item  Let $n$ and $k$ be positive integers with $k\geq n$ and $k-n$ an even number. Let $2n$ lamps labelled $1, 2,\dots, 2n$ be given, each of which can be either on or off. Initially all the lamps are off.\\

	We consider sequences of steps: at each step one of the lamps is switched ( from on to off or from offto on).\\

Let $N$ be the number of such sequences consisting of $k$ steps and resulting in the state where lamps $1$ through $n$ are all on, and lamps $n + 1$ through $2n$ are all off.\\

Let $M$ be the number of such sequences consisting of $k$ steps, resulting in the state where lamps $1$ through $n$ are all on, and lamps $n + 1$ through $2n$ are all off, but where none of the lamps $n + 1$ through $2n$ is ever switched on.\\
	Determine the ratio $\frac{N}{M}$.

\item Let $ABCD$ be a convex quadrilateral with $|BA| \neq |BC|$.Denote the incircles of triangles $ABC and ADC$ by $\omega_1 and \omega_2$ respectively.\\
Suppose that there exists a circle $\omega$ tangent to the ray $BA$ beyond $A$ and tothe ray $BC$ beyond $C$, which is also tangent to the lines $AD$ and $CD$.Prove that the common external tangents of $\omega_1$ and $\omega_2$ intersect on $\omega$.
\end{enumerate}
\end{document}
